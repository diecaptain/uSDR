\subsection{max2831}
The following functions are provided in the max2831 driver. Although the MAX2831 internal registers'
cannot be read through SPI interface, the driver maintains the registers' values. User can retrieves
these value through a provided function.

\begin{enumerate}
	\item void initialize\_chip();
	\item void setToDefaultRX(int register\_number);
	\item void setRegisterValueRX(int register\_number,uint32\_t register\_value);
	\item uint32\_t getRegisterValueRX(int reg\_number);
	\item uint8\_t setLNAGain(uint8\_t LNAGain);
	\item uint8\_t setRXBaseBandVGAGain(uint8\_t VGAGain);
	\item uint8\_t setTXBaseBandVGAGain(uint8\_t VGAGain);
	\item void enRXParVGAGainCtl();
	\item uint8\_t setFreqDivider(int freq);
	\item int getFreqDivider();
	\item uint8\_t setBaseBandLowPassFilterMode (uint8\_t LPFMode);
	\item uint8\_t setTXBaseBandLowPassFilterModeFineAdjust (uint8\_t LPFMode, uint8\_t LPFFineAdjust);
	\item uint8\_t setRXBaseBandLowPassFilterModeFineAdjust (uint8\_t LPFMode, uint8\_t LPFFineAdjust);
\end{enumerate}

\subsubsection{void initialize\_chip()}
\hangindent=\parindent
\hangafter=0
Input argument: None\\
Return value: None\\
This function resets all registers to the default value.

\subsubsection{void setToDefaultRX(int register\_number)}
\hangindent=\parindent
\hangafter=0
Input argument: Register number. A valid register number is from 0 to 15.\\
Return value: None\\
This function sets the specified register to its default value.

\subsubsection{void setRegisterValueRX(int register\_number,uint32\_t register\_value)}
\hangindent=\parindent
\hangafter=0
Input argument: Register number, Register value. A valid register number is from 0 to 15.\\
Return value: None\\
This function set the specified register to specified value.

\subsubsection{uint32\_t getRegisterValueRX(int reg\_number)}
\hangindent=\parindent
\hangafter=0
Input argument: Register number. A valid register number is from 0 to 15.\\
Return value: Register value\\
This function returns the specified register's value.

\subsubsection{uint8\_t setLNAGain(uint8\_t LNAGain)}
\hangindent=\parindent
\hangafter=0
Input argument: LNA gain. A valid LNA gain is from 0 to 3.\\
Return value: Success/Fail. 0: Fail, 1: Success.\\
This function sets the LNA gains in RX loop and return success/fail.
\begin{table}
\centering
	\begin{tabular}{|c|c|}
		\hline
		{\bf LNA Gain} & {\bf Mode}\\ \hline
		0 or 1 & Low gain\\ \hline
		2 & Medium gain\\ \hline
		3 & High gain\\ \hline
	\end{tabular}
	\caption{LNA gain mode}
\end{table}

\subsubsection{uint8\_t setRXBaseBandVGAGain(uint8\_t VGAGain)}
\hangindent=\parindent
\hangafter=0
Input argument: VGA gain. A valid VGA gain is from 0 (min) to 31 (max). Step size = 2~dB\\
Return value: Success/Fail. 0: Fail, 1: Success.\\
This function sets the {\bf RX} baseband gain.

\subsubsection{uint8\_t setTXBaseBandVGAGain(uint8\_t VGAGain)}
\hangindent=\parindent
\hangafter=0
Input argument: VGA gain. A valid VGA gain is from 0 (min) to 63 (max). Step size = 2~dB\\
Return value: Success/Fail. 0: Fail, 1: Success.\\
This function set the {\bf TX} baseband gain.

\subsubsection{void enRXParVGAGainCtl()}
\hangindent=\parindent
\hangafter=0
Input argument: None\\
Return value: None\\
This function enables the parallel control over the RX gain. It's necessary for AGC and Only 
valid on \sdr v2.

\subsubsection{uint8\_t setFreqDivider(int freq)}
\hangindent=\parindent
\hangafter=0
Input argument: RF Frequency\\
Return value: Success/Fail. 0: Fail, 1: Success.\\
This function sets the RF frequency of MAX2831.

\subsubsection{int getFreqDivider()}
\hangindent=\parindent
\hangafter=0
Input argument: None\\
Return value: RF frequency
This function returns the RF frequency of MAX2831.

\begin{comment}
\subsubsection{uint8\_t setBaseBandLowPassFilterMode (uint8\_t LPFMode)}
\hangindent=\parindent
\hangafter=0
Input argument: Low Pass Filter Mode\\
Return value: Success/Fail. 0: Fail, 1: Success.\\
\end{comment}

\subsubsection{uint8\_t setTXBaseBandLowPassFilterModeFineAdjust (uint8\_t LPFMode, uint8\_t LPFFineAdjust)}
\hangindent=\parindent
\hangafter=0
Input argument: Low Pass Filter Mode, Adjustment\\
Return value: Success/Fail. 0: Fail, 1: Success.\\
See table~\ref{tab:lpf} for more details.

\subsubsection{uint8\_t setRXBaseBandLowPassFilterModeFineAdjust (uint8\_t LPFMode, uint8\_t LPFFineAdjust)}
\hangindent=\parindent
\hangafter=0
Input argument: Low Pass Filter Mode, Adjustment\\
Return value: Success/Fail. 0: Fail, 1: Success.\\
See table~\ref{tab:lpf} for more details.

\begin{table}[h]
\centering
	\begin{tabular}{|c|c|c|c|c|c|c|c|c|c|c|c|}
	\hline
	\multirow{3}{*}{LPF Mode} &
	\multicolumn{11}{c|}{LPF Fine Adjustment}\\ \cline{2-12}
	& \multicolumn{5}{c|}{RX} &
	\multicolumn{6}{c|}{TX} \\ \cline{2-12}
			& {\bf 0} & {\bf 1} & {\bf 2} & {\bf 3} & {\bf 4} & {\bf 0} & {\bf 1} & {\bf 2} & {\bf 3} & {\bf 4} & {\bf 5}\\ \hline
	{\bf 0} & \multirow{4}{*}{90\%} & \multirow{4}{*}{95\%} & 7.5~MHz & \multirow{4}{*}{105\%} & \multirow{4}{*}{110\%} 
			& \multirow{4}{*}{90\%} & \multirow{4}{*}{95\%} & 8~MHz   & \multirow{4}{*}{105\%} & \multirow{4}{*}{110\%} & \multirow{4}{*}{115\%}\\ \cline{1-1}\cline{4-4}\cline{9-9}
	{\bf 1} &  &  & 8.5~MHz &  &  &  &  & 11~MHz	&  &  & \\ \cline{1-1}\cline{4-4}\cline{9-9}
	{\bf 2} &  &  & 15~MHz 	&  &  &  &  & 16.5~MHz	&  &  & \\ \cline{1-1}\cline{4-4}\cline{9-9}
	{\bf 3} &  &  & 18~MHz 	&  &  &  &  & 22.5~MHz	&  &  & \\ \hline
	\end{tabular}
	\caption{LPF adjustment details}
	\label{tab:lpf}
\end{table}

