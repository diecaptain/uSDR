\chapter{Hardware}

\hl{NOTE: needs a top level diagram. That would make the connections a lot easier to follow.}

\hl{IDEA: combine all of the IO connection tables into one super table so all wiring is in one spot.}

The major components of \sdr are the following:
\begin{description}
	\item[Microsemi Smartfusion] \hfill \\
	The procssing core of \sdr. It incorporates an
	analog computing engine, a flash-based FPGA and a hardcore ARM Cortex-M3.
	\item[MAXIM MAX2831] \hfill \\
	The MAXIM 2831 is a 2.4~GHz ISM band RF frontend. This
	chip integrates common RF blocks (power amplifier, filters, voltage
	control oscillator, etc.) into a single chip.
	\item[Analog Device AD9288] \hfill \\
	8-bits Dual channel high speed ADC. Three variants
	are available. The one which is currently employed on \sdr is 80~MSP/s.
	\item[MAXIM MAX5189] \hfill \\
	8-bits Dual channel 40~MHz DAC. Both channels share
	the same input but latching the data using different edges of clock input.
	\item[Micrel KS8721CL] \hfill \\
	100Base-TX Physical Layer Transceiver providing RMII interfaces to MACs
	and switches.
	\item[FTDI FT232R] \hfill \\
	USB to UART interface. It connects to the UART0 of SmartFusion device.
	\item[External Mem] \hfill \\
	\sdr has 2 external memories which are 16~MB PSRAM and 8~MB flash installed.

\end{description}

\section{Power}
\subsection{Power Rails}
\sdr uses many different power rails for many purposes. Every power rail is produced by an
LDO (Low Drop-Out regulator).
\begin{enumerate}
	\item 3.3~V, the major power supply for {\bf Smartfusion, ADC, DAC, RF, Ethernet,
	USB, RTC, PSRAM, and FLASH}
	\item 2.85~V, {\bf RF} only.
	\item 1.8~V, {\bf PSRAM} only.
	\item 1.5~V, {\bf Smartfusion(Core)} only.
\end{enumerate}

\subsection{Power Source}
\sdr can be powered by USB, PoE, exposed headers or power jack. A 4-way switch is used to select
the input power source.

\subsection{1.5V on-chip regulator}
The SmartFusion integrates a 1.5~V on-chip regulator. This regulator could be turned on
by triggering a specific IO or software interrupt. Similarly, this regulator could
be turned off by setting a specific register.

Jumper P15 (``int1.5\_en'') must be jumpered in order to enable the on-chip 1.5~V regulator. Jumper P16 (``V\_1.5int'')
is the 1.5~V power selector. Jumpering the upper 2 pins powers the \sdr using the external
regulator. In contrast, jumpering the bottom 2 pins causes \sdr to use the on-chip 1.5~V regulator.

\subsection{Enabling/Disabling Power}
\sdr offers many jumpers to separate the power domain across subsystems. User can
manually disconnect the power supplies by removing the jumpers. These jumpers also provide
convenient locations to measure the current
through a specific chip.

Note: 1.5~V and 3.3~V on SmartFusion and Ethernet are required
for the \sdr to operate properly.

\hl{NOTE: this needs to be further explained with a table of what jumpers power what.}

\section{RF Frontend}
The MAX2831 provides a 3-wire SPI interface as well as a 7-bit wide gain control bus.
Every register can be written through the SPI interface, however, the SPI interface doesn't allow
for reading the current registers' value. TX baseband gain, RX baseband gain and
RX LNA gain can be controlled using these registers and 7-bit bus. Since the SPI interface clocks
a single bit at a time, it has longer latency when setting the gain control than the 7-bit bus.
Therefore, AGC benefits from the parallel bus.
In order to enable the parallel bus gain control, registers 8 D12 and 9 D10
must be asserted. In the receiving mode, bits B7 and B6 control the LNA gain setting and bits B5\~{}B1
control the baseband VGA gain.

Two additional control wires (RX\textbackslash{}TX, SHDN) control the modes. Available
modes are listed in Table~31 in datasheet~\cite{MAX2831}.

The analog RSSI signal is connected to a low-power serial ADC~\cite{ADC081S101} that operates up to 1~MSP/s.
The maximum output of serial ADC is 223 (3.3~V) instead of 255. Therefore, the step size
of serial ADC is {\bf 14.8~mV}.

The MAX2831 separates the TX/RX path. Since it cannot fully duplex communication, an
RF switch~\cite{AS213} is used to allow a single antenna. If desired, users can
use separate antennas by applying the three small changes listed below.

{\bf Instructions for Using Separate Antennas:}
\begin{enumerate}[noitemsep]
	\item Remove C184, C186.
	\item Short R94, R95.
	\item Install antennas on J31, J33.
\end{enumerate}

\subsection{Test Points}
Various test points are available on \sdr. Analog signals include TX/RX I/Q channels, LD, CLK and RSSI.
P12, P13 (on the edge of \sdr) provide digital test points for SPI interface of MAX2831 and
ADC081S101.

\hl{NOTE: need a better table for this. It's not clear which pins exactly
correspond to which SPI signals.}

\subsection{IO connection}
MAX2831 interface
\begin{table}[h]
\centering
\begin{tabular}{|l|c|c|}
	\hline
	{\bf Name} & {\bf Physical Connection} & {\bf Direction}\\ \hline
	RX\textbackslash{}TX 	& L18 & Out\\ \hline
	SHDN 	& M18 & Out\\ \hline\hline
	\={CS}	& H20 & Out\\ \hline
	DATA	& J22 & In\\ \hline
	CLK		& L22 & Out\\ \hline\hline
	B1 (VGA, LSB) 	& D7 & Out\\ \hline
	B2 (VGA)		& E8 & Out\\ \hline
	B3 (VGA)		& C4 & Out\\ \hline
	B4 (VGA)		& C5 & Out\\ \hline
	B5 (VGA, MSB)	& D8 & Out\\ \hline
	B6 (LNA, LSB)	& C7 & Out\\ \hline
	B7 (LNA, MSB)	& C8 & Out\\ \hline
\end{tabular}
\end{table}

RSSI Serial ADC (ADC081S101) interface (SPI)
\begin{table}[h]
\centering
\begin{tabular}{|l|c|c|}
	\hline
	{\bf Name} & {\bf Physical Connection} & {\bf Direction}\\ \hline
	\={CS}	& H17 & Out\\ \hline
	DATA	& H22 & In\\ \hline
	CLK		& H18 & Out\\ \hline
\end{tabular}
\end{table}

\clearpage
\section{ADC}
Channels A and B of the AD9288~\cite{AD9288} are connected to the In-phase and Quad-phase channels of the MAX2831, respectively.
The input common-mode voltage of the AD9288 is 0.3$*$VDD which is 0.99~V. However, the common-mode
voltage of the MAX2831 is not able to be tuned to 0.99~V. A voltage reference chip, the MAX6061~\cite{MAX6061},
and a voltage divider are used to alter the common-mode voltage.

The output of the AD9288 on \sdr is pre-configured to 2's complement output. Therefore,
D7\subscript{A, B} are sign bits for each channel. Deassert signal S1 to put the AD9288 into
standby mode. More detail can be found in Table~4 in datasheet~\cite{AD9288}.

\subsection{IO connection}
\begin{table}[h]
\centering
\begin{tabular}{|l|c|c|}
	\hline
	{\bf Name} & {\bf Physical Connection} & {\bf Direction}\\ \hline
	D0\subscript{A} (I, LSB) 	& C21 & In\\ \hline
	D1\subscript{A} 			& D21 & In\\ \hline
	D2\subscript{A}				& B20 & In\\ \hline
	D3\subscript{A}				& C19 & In\\ \hline
	D4\subscript{A}				& G19 & In\\ \hline
	D5\subscript{A}				& F19 & In\\ \hline
	D6\subscript{A}	(I, MSB)	& G21 & In\\ \hline
	D7\subscript{A} (I, Sign)	& G20 & In\\ \hline\hline
	D0\subscript{B} (Q, LSB) 	& K17 & In\\ \hline
	D1\subscript{B} 			& J17 & In\\ \hline
	D2\subscript{B}				& F21 & In\\ \hline
	D3\subscript{B}				& F20 & In\\ \hline
	D4\subscript{B}				& G18 & In\\ \hline
	D5\subscript{B}				& G17 & In\\ \hline
	D6\subscript{B}	(Q, MSB)	& E18 & In\\ \hline
	D7\subscript{B} (Q, Sign)	& F17 & In\\ \hline\hline
	CLK							& A17 & Out\\ \hline\hline
	S1							& D18 & Out\\ \hline
\end{tabular}
\end{table}

\clearpage
\section{DAC}
Similarly, the MAX6061 voltage reference chip is deployed on the transmission path to pull the
common-mode voltage. The MAX5189~\cite{MAX5189} uses both edges of the clock to update both
channel outputs. The maximum clock input frequency of the MAX5189 is 40~MHz. The combination
of DACEN and PD determines the operation mode, more detailed information can be found
in Table~1 of the datasheet~\cite{MAX5189}.

\subsection{IO connection}
\begin{table}[h]
\centering
\begin{tabular}{|l|c|c|}
	\hline
	{\bf Name} & {\bf Physical Connection} & {\bf Direction}\\ \hline
	D0 (LSB)	& E1  & Out\\ \hline
	D1			& F3  & Out\\ \hline
	D2			& G4  & Out\\ \hline
	D3			& H5  & Out\\ \hline
	D4			& H6  & Out\\ \hline
	D5			& J6  & Out\\ \hline
	D6			& B22 & Out\\ \hline
	D7 (MSB)	& C22 & Out\\ \hline\hline
	CLK			& A18 & Out\\ \hline\hline
	DACEN		& F1  & Out\\ \hline\hline
	PD			& G2  & Out\\ \hline
\end{tabular}
\end{table}

\section{User's IO}
\sdr has eight user-defined LEDs and four push-button switches. The LEDs and switches are {\bf active low}. The LEDs/switches
are connected to the Fabric, however, users can route the connection to the MSS IO. In addition, two IO
banks also available on \sdr. A 16-pin wide interface is connected to the Fabric, whereas an 8-pin
bus is connected to fixed\footnote{GPIOs on Fabric are able to route to MSS IO Pads, however,
GPIOs on MSS cannot be routed to Fabric.}
MSS GPIOs. Libero SoC must be configured to enable the MSS GPIO.

\subsection{IO connection}
\begin{table}[h]
\centering
\begin{tabular}{|l|c|c|}
	\hline
	{\bf Name} & {\bf Physical Connection} & {\bf Direction}\\ \hline
	LED0 & J20 & Out \\ \hline
	LED1 & J19 & Out \\ \hline
	LED2 & K20 & Out \\ \hline
	LED3 & K21 & Out \\ \hline
	LED4 & L20 & Out \\ \hline
	LED5 & L21 & Out \\ \hline
	LED6 & K18 & Out \\ \hline
	LED7 & K19 & Out \\ \hline\hline
	SW1  & J21 & In	 \\ \hline
	SW2  & B19 & In	 \\ \hline
	SW3  & D15 & In	 \\ \hline
	SW4  & E14 & In	 \\ \hline
\end{tabular}
\end{table}

\clearpage

\begin{table}[h]
\centering
\begin{tabular}{|l|c|c|}
	\hline
	{\bf Name} & {\bf Physical Connection} & {\bf Direction}\\ \hline
	Fabric IO\subscript{0}	& N6 & In/Out \\ \hline
	Fabric IO\subscript{1}	& M6 & In/Out \\ \hline
	Fabric IO\subscript{2}	& P1 & In/Out \\ \hline
	Fabric IO\subscript{3}	& P2 & In/Out \\ \hline
	Fabric IO\subscript{4}	& N3 & In/Out \\ \hline
	Fabric IO\subscript{5}	& N2 & In/Out \\ \hline
	Fabric IO\subscript{6}	& M2 & In/Out \\ \hline
	Fabric IO\subscript{7}	& M1 & In/Out \\ \hline
	Fabric IO\subscript{8}	& M4 & In/Out \\ \hline
	Fabric IO\subscript{9}	& L3 & In/Out \\ \hline
	Fabric IO\subscript{10}	& L2 & In/Out \\ \hline
	Fabric IO\subscript{11}	& L1 & In/Out \\ \hline
	Fabric IO\subscript{12}	& L5 & In/Out \\ \hline
	Fabric IO\subscript{13}	& K4 & In/Out \\ \hline
	Fabric IO\subscript{14}	& L6 & In/Out \\ \hline
	Fabric IO\subscript{15}	& K6 & In/Out \\ \hline\hline
	MSS IO\subscript{0}		& T3 & In/Out \\ \hline
	MSS IO\subscript{1}		& V3 & In/Out \\ \hline
	MSS IO\subscript{2}		& U3 & In/Out \\ \hline
	MSS IO\subscript{3}		& T4 & In/Out \\ \hline
	MSS IO\subscript{4}		& AA2& In/Out \\ \hline
	MSS IO\subscript{5}		& AB2& In/Out \\ \hline
	MSS IO\subscript{6}		& AB3& In/Out \\ \hline
	MSS IO\subscript{7}		& Y3 & In/Out \\ \hline
\end{tabular}
\end{table}

\section{TCXO}
The PCF2127A~\cite{PCF2127A} is a temperature-compensated low-frequency oscillator. It provides $\pm$3ppm stability
over a wide range of temperature and it replaces the 32.768~KHz crystal oscillator \hl{ replaces what crystal oscillator?}.
The configuration
interface is connected to the MSS SPI bus 1\footnote{It's a fixed connection and needs to be initiated in
Libero SoC MSS configuration tool.} and the interrupt is connected to a Fabric IO. A external coin-cell
battery provides an additional power source for the PCF2127A. \sdr automatically switches the power source if V\subscript{DD} is
less than a certain threshold. \hl{ Less than what threshold?}
\subsection{IO connection}
\begin{table}[h]
\centering
\begin{tabular}{|l|c|c|}
	\hline
	{\bf Name} & {\bf Physical Connection} & {\bf Direction}\\ \hline
	INT & B7 & In\\ \hline\hline
	D\subscript{IN} & T17 & Out\\ \hline
	D\subscript{OUT} & V19 & In\\ \hline
	CLK & AA22 & Out\\ \hline
	\={CS} & W21 & Out\\ \hline
\end{tabular}
\end{table}

\clearpage
\section{Ethernet}
\subsection{Power over Ethernet}
\sdr can be powered by Ethernet. The PoE module on the \sdr is the AG9205S~\cite{AG9200S} and it provides up to
13~W power output. The default power setting on \sdr is configured to 3.84~W, which is equivalent to 768~mA
maximum current. The setting can be changed by replacing the configuration
resistor R50. More power ratings are available in Table~2 in datasheet~\cite{AG9200S}.

\subsection{Disabling KSZ8721}
The Ethernet transceiver KSZ8721 on \sdr is enabled by default. The transceiver draws about 50~mA
when in an idle state. In low-power applications when the transceiver is not needed, it can be turned off\footnote{
The Ethernet subsystem cannot be disabled by removing the jumper P7 (``V\_eth3.3''). Doing this results in the SmartFusion working
improperly.} by patching the \sdr.

{\bf Instructions to Disable the KSZ8721 Ethernet Module:}
\begin{enumerate}[noitemsep]
	\item Remove R36.
	\item Put 10~K resistor on R33.
\end{enumerate}

\subsection{IO connection}
\begin{table}[h]
\centering
\begin{tabular}{|l|c|c|}
	\hline
	{\bf Name} & {\bf Physical Connection} & {\bf Direction}\\ \hline
	CLK (50~MHz) & E3 & In \\ \hline
\end{tabular}
\end{table}

%\section{USB}
%\section{External Memory}
