\clearpage
\subsection{radio\_config}
This driver provides the low-level radio hardware configuration and source address information.
The radio structure below shows its data member. All the data member are address correlated.
User has to note that the fabric doesn't perform the address recognition because of the area
constraint. In order for the radio to operate properly, user has to perform address recoginition
in time. In addition, multiple source addresses is supported by software configuration. 

\begin{lstlisting}[caption=Radio Structure]
struct radio{
	uint8_t multiple_addr_en;
	uint8_t num_of_multiple_addr;
	uint32_t* multiple_addr_ptr;
	uint16_t src_pan_ID;
	uint8_t src_addr_mode;
	uint32_t src_addr_LSB, src_addr_MSB;
};
\end{lstlisting}

\begin{enumerate}
	\item void init\_system();
	\item inline void tx\_fire();
	\item void RF\_shdn(uint8\_t shdn);
	\item inline void auto\_ack\_en(uint8\_t ack\_en);
	\item inline void rx\_agc\_en(uint8\_t agc\_mode);
	\item inline void ack\_flush();
	\item inline void auto\_tx\_en(uint8\_t en);
	\item uint8\_t get\_radio\_mode();
	\item uint8\_t dest\_addr\_filter(struct rx\_packet\_str* rp, struct radio* r0);
	\item void set\_multiple\_address(struct radio* r0, uint8\_t en, uint8\_t nu, uint32\_t* map);
	\item inline uint8\_t set\_src\_addr\_mode(struct radio* r0, uint8\_t sam);
	\item inline void set\_src\_pan(struct radio* r0, uint16\_t sp);
	\item inline void set\_src\_addr(struct radio* r0, uint32\_t sa1, uint32\_t sa0);
\end{enumerate}


\subsubsection{void init\_system()}
\hangindent=\parindent
\hangafter=0
Input argument: None\\
Return value: None\\
This function initialize the entire radio including MAX2831, GPIO and PDMA drivers. This function
should be called before any other radio correlated functions.

\subsubsection{inline void tx\_fire()}
\hangindent=\parindent
\hangafter=0
Input argument: None\\
Return value: None\\
This function starts transmitting the content of TX fifo.

\subsubsection{void RF\_shdn(uint8\_t shdn)}
\hangindent=\parindent
\hangafter=0
Input argument: radio shutdown\\
Return value: None\\
This function turns off all the radio peripherals. (RF, ADCs and DAC)

\subsubsection{inline void auto\_ack\_en(uint8\_t ack\_en)}
\hangindent=\parindent
\hangafter=0
Input argument: ACK enable/disable\\
Return value: None\\
\begin{table}[h]
\centering
	\begin{tabular}{|c|c|}
	\hline
		{\bf Auto ACK en (ack\_en)} & {\bf Effect}\\ \hline
		0 & Disable Auto ACK\\ \hline
		1 & Enable Auto ACK\\ \hline
	\end{tabular}
	\caption{ACK Enable/Disable}
\end{table}


\subsubsection{inline void rx\_agc\_en(uint8\_t agc\_mode)}
\hangindent=\parindent
\hangafter=0
Input argument: AGC mode\\
Return value: None\\
\begin{table}[h]
\centering
	\begin{tabular}{|c|c|}
	\hline
		{\bf AGC mode (agc\_mode)} & {\bf Effect}\\ \hline
		0 & Disable AGC\\ \hline
		1 & SFD-Latch AGC\\ \hline
		2 & Reserved\\ \hline
		3 & Continues AGC\\ \hline
	\end{tabular}
	\caption{AGC Mode}
\end{table}

\subsubsection{inline void ack\_flush()}
\hangindent=\parindent
\hangafter=0
Input argument: None\\
Return value: None\\
This function flushes the outgoing ACK packet. Once a packet reception completed, user should
unload the packet from the radio and perform destination address filter. It returns whether
the address is match to radio's source address or not. In the mean time, user needs to check
the coming packet has ACK request bit set or not. If the following conditions are met, user
should call ack\_flush();
\begin{enumerate}
	\item Destination address {\bf doesn't} match to local source address.
	\item Packet passes the CRC check.
	\item Packet has ACK request field set.
\end{enumerate}

\subsubsection{inline void auto\_tx\_en(uint8\_t en)}
\hangindent=\parindent
\hangafter=0
Input argument: Auto TX enable\\
Return value: None\\
This function enables the automatic transmission of TX fifo. User can achieve concurrent
transmission using this function.
\begin{table}[h]
\centering
	\begin{tabular}{|c|c|}
	\hline
		{\bf Auto TX enable (en)} & {\bf Effect}\\ \hline
		0 & Disable Auto TX\\ \hline
		1 & Radio starts transmitting after 192~$\mu$s of any packet received\\ \hline
	\end{tabular}
	\caption{Auto TX}
\end{table}

\subsubsection{uint8\_t get\_radio\_mode()}
\hangindent=\parindent
\hangafter=0
Input argument: None\\
Return value: Radio mode\\
This function is mainly used for debugging purpose.
\begin{table}[h]
\centering
	\begin{tabular}{|c|c|c|}
	\hline
		{\bf Radio Mode} & {\bf Fabric State} & {\bf Corresponding State}\\ \hline
		0x00 & RX\_IDLE & Idle Listening\\ \hline
		0x01 & RX\_COLLECT & RX\\ \hline
		0x03 & TX\_RX\_TURNAROUND & \\ \hline
		0x04 & RX\_TX\_TURNAROUND & \\ \hline
		0x05 & RX\_WAIT\_FOR\_ACK\_GLOSSY & \\ \hline
		0x08 & RADIO\_OFF & Off\\ \hline
		0x09 & RADIO\_WARMUP & \\ \hline
		0x0c & WAIT\_FOR\_TX\_COMPLETE & TX\\ \hline
		0x0d & RX\_WAIT\_FOR\_ACK\_GLOSSY\_COMPLETE & \\ \hline
	\end{tabular}
	\caption{Radio Mode}
\end{table}


\subsubsection{uint8\_t dest\_addr\_filter(struct rx\_packet\_str* rp, struct radio* r0)}
\hangindent=\parindent
\hangafter=0
Input argument: rx\_packet\_str pointer, radio struct pointer\\
Return value: Match/Mismatch. 0: Mismatch, 1: Match.\\

\subsubsection{void set\_multiple\_address(struct radio* r0, uint8\_t en, uint8\_t nu, uint32\_t* map)}
\hangindent=\parindent
\hangafter=0
Input argument: radio struct pointer, multiple address enable, 
number of multiple address, multiple address array\\
Return value: None\\

\subsubsection{inline uint8\_t set\_src\_addr\_mode(struct radio* r0, uint8\_t sam)}
\hangindent=\parindent
\hangafter=0
Input argument: radio struct pointer, Source Address Mode (sam)\\
Return value: Success/Fail. 0: Fail, 1: Success.\\
See table~\ref{tab:address_mode} for detail information.

\subsubsection{inline void set\_src\_pan(struct radio* r0, uint16\_t sp)}
\hangindent=\parindent
\hangafter=0
Input argument: radio struct pointer, Source PAN ID (sp)\\
Return value: None\\
This function sets the source PAN ID in a radio struct.

\subsubsection{inline void set\_src\_addr(struct radio* r0, uint32\_t sa1, uint32\_t sa0)}
\hangindent=\parindent
\hangafter=0
Input argument: radio struct pointer, Source Address (sa1, sa0)\\
Return value: None\\
This function sets the source address in a radio struct. Source address could be as
long as 8 bytes (source address mode = 3). If 8 bytes mode is selected, source  
address = {sa1 (higher 32-bits), sa0 (lower 32-bits)}. Similarly, source address = 
(sa0 \& 0xffff) if 2 bytes mode is selected.

