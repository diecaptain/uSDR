% ABSTRACT

Most modern software-defined radios are large, expensive, and
power-hungry devices, and this hampers their deployment and use in
low-power, size-constrained settings like sensor networks and mobile
computing.  We explore the viability of scaling down the software
radio in size, cost, and power, and show that an index card-sized,
sub-\$150, `AA' battery-powered system is possible using off-the-shelf
components.  Key to our approach is that we leverage an integrated,
reconfigurable, flash-based FPGA with a hard ARM Cortex-M3
microprocessor which simultaneously enables lower power and tighter
hardware/soft\-ware integration than prior software radios.  This
architecture allows us to realize the first sub-watt software radio 
paltform, implement timing-critical MAC protocols, and
validate the speculated performance of several recent MAC/PHY
primitives and protocols (\textit{e.g.} Backcast, A-MAC, and Glossy) using
an IEEE 802.15.4-compliant radio implementation.
\begin{comment}
The work also identifies several improvements
in the underlying hardware components that could improve power,
performance, and flexibility.
\end{comment}
