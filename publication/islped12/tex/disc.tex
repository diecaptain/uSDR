\section{Discussion}
\label{sec:disc}
\begin{comment}
We built \sdr with the intention to see how close we can get to the claims
presented in~\cite{dutta-low-cal}. This section will look at the four
different requirements {\em Radio Duty-Cycling}, \emph{Low-Power FPGA},
\emph{System Integration}, and \emph{Measurement} and compare the current \sdr
platform to the performance it is supposed to achieve in order to compete with
current commercially available low-power wireless radios.
\end{comment}

This section looks at the radio requirements claimed in prior work
and comparing the current \sdr
to the performance it is supposed to achieve in order to compete with
current commercially available low-power wireless radios.

\subsection{Radio Duty-Cycling}

Typical duty-cycled radio platforms achieve $<$1~mW sleep power, wake up in
$\mu$s, and draw tens to hundreds of mW while actively using the radio. \sdr
draws in it's lowest sleep state 322~mW (wakeup in 3.01~ms) or 518~mW (wakeup in
34~$\mu$s). During transmit, \sdr draws 1.4~W, and 1~W while receiving.
While the sleep current is still three orders of magnitude greater than typical
duty-cycled systems, it is a good start to explore the space of battery
powered SDR platforms since nodes can last for many hours on `AA' batteries. 

\begin{comment}
Comparing this to an iPhone~4S which has a battery
capacity of 5.3~Whr, we can power \sdr for 3.8~hours while constantly
transmitting, or 16.4~hours in deep sleep. Thus, using a moderate amount of
duty-cycling, a 12~hour deployment time could easily be achieved.
\end{comment}

\subsection{Low-Power FPGA}

During sleep mode, radio frontend,
ADC, and DAC support sleep modes in the $\mu$W range.
Still, \sdr draws hundreds of mW. Unfortunately, the SmartFusion
does not include the latest Flash*Freeze technology, allowing to clock-gate
the FPGA itself. Thus, even in deep sleep, the FPGA is still running and
wasting energy. We expect that future iterations of the SmartFusion will
include such technology.

\begin{comment}
\subsection{System Integration}

\sdr significantly departed from the common platform reconfigurability of
typical SDR systems, integrating everything onto a 91~cm$^\mathrm{2}$ sized PCB. This
reduced cost to $<$\$150. While we didn't achieve the expected \$100 suggested
in \cite{dutta-low-cal}, we got really close. We expect that with larger
quantities and the commoditization of these mixed-signal FPGAs with
hard-silicon cores, the total price would fall and drop below the magical
\$100 mark.

\subsection{Measurement}

The current iteration of the \sdr platform does not include any energy
metering capabilities that are exposed to the application processor. While we
can measure the external power draw on the different power rails, there are no
components of the SmartFusion connected to meter those rails internally.
However, the SmartFusion has a full Analog Compute Engine with up to 12 direct ADC
inputs. The current \sdr platform does not take advantage of any of them,
and we plan to add this support in the next iteration of the platform.
\end{comment}
