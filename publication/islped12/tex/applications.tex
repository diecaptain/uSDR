\section{Applications}
\label{sec:applications}

\begin{table}[b]
\centering
\begin{threeparttable}
\begin{tabularx}{\columnwidth}{|X|X|X|}
    \rowcolor[gray]{0}
   {\sc {\color{white} Component}}
 & {\sc {\color{white} Area}}
 & {\sc {\color{white} Max Freq}}
\\\hline
TX & 20\% & 100 MHz \\\hline
RX & 72\%\tnote{a} & 80 MHz\tnote{b} \\\hline
\end{tabularx}
\begin{tablenotes}
	\small
	\item [a] Area optimized for minimum required frequency of 16~MHz.
	\item [b] Speed optimized increases area to 78\%.
\end{tablenotes}
\caption{FPGA size and corresponding maximum frequency. The area with
respect to the 500k gate SmartFusion A2F500. The transmitter is limited by
80~MHz DAC, while the receiver is FPGA limited.}
\label{tab:area}
\end{threeparttable}
\end{table} 
This section presents the viability of using the \sdr platform to communicate
with standards compliant low-power radios. We decided to implement IEEE
802.15.4, a non-trivial physical communication specification used in many
short-range, low-power radios. 

The short-range low-power wireless standard IEEE 802.15.4 describes the
physical layer and frame structure, leaving the higher layers up to the
developer to design. IEEE 802.15.4 uses spread spectrum technology to increase
its resilience towards channel noise. Every 4-bit data sequence gets mapped to
a 32-bit long chip sequence. This chip sequence is then modulated as an
offset quadrature phase-shift keying (O-QPSK) signal with half-sine pulses. The sixteen
32-bit chip sequences are chosen such that their autocorrelation is minimized.
This allows a receiver to compare incoming chip sequences to one of the sixteen
existing sequences, allowing for errors within the received
sequence, but resulting in a successful sequence detection with high
probability.

IEEE 802.15.4 is a packet-based standard. Each packet, or frame, consists of a
pre-defined sequence. At first, every packet starts with a preamble of four
bytes of 0x00. This is followed by the Start of Frame Delimiter (SFD) byte
0xA7. The preamble is used to condition a radio and detect an incoming packet,
while the SFD signals the beginning of the data. The first data byte has to be
the length of the rest of the packet. Following the length byte comes the
Frame Control Field (FCF). The FCF is a two byte value indicating different
properties of the message, such as addressing mode (length of addresses used),
is an acknowledgment requested, etc. Following the FCF comes a one byte
sequence number (Data Sequence Number, DSN) followed by the address
information. For data integrity, IEEE 802.15.4 uses a two byte long Cyclic Redundancy Check
(CRC). The CRC is added to the very end of a message.

Figure \ref{fig:system_arch} shows the general \sdr architecture. For our IEEE
802.15.4 implementation, we implemented a transmitter with O-QPSK modulator
representing the transmit path. For the receiver we exploit the fact that
O-QPSK with half-sine pulse shapes is equivalent to minimum shift keying
(MSK), and thus we use an FM demodulator. The following sections go into
details of the specific implementations. To exemplify the potential of the
\sdr platform, we describe in Section \ref{subsec:enhance} enhancements we
performed that usually are not found on commercial radios, but support recent
advances in wireless protocol designs.

%Different SDR platforms target different design spaces, ranging from
%full-software inmplementation on a general purpose processor like in GNU
%Radio or MSR's SORA, to radios with highly reconfigurable hardware like the
%WARP. \sdr explores the architecture trade-off between hardware and software
%to build a SDR platform which is well suited for mobile, low-power
%applications. Compared to GNU Radio or MSR's SORA which use a commodity PC as
%baseband processor, \sdr has limited computing resources available. Therefore,
%\sdr implements the basic radio functions in the reconfigurable hardware.
%While this puts a bigger burden on the developer, as the radio blocks have to
%be implemented in a Hardware Description Language (HDL), it makes the system
%resemble regular radios, as the interface between software and hardware
%are memory-mapped registers. 

\subsection{Transmitter}

We implemented the full transceiver chain as a memory-mapped peripheral on the
FPGA. This reduces the burden and complexity of the software. To send out a
message, the core transfers the packet from memory into a transmit fifo (see
Figure \ref{fig:transmitting_chain}). The processor can start the transmission
before the full packet is uploaded by sending a transmit command to the
peripheral. The peripheral starts encoding the packet, starting with the
preamble, SFD, and FCF, before adding the bytes from the transmit fifo. The
bytes go through several encoding steps, translating from half-bytes (4-bit) to
chip sequence (32-bit), and finally getting mapped to an O-QPSK signal with
half-sine pulse shapes on the in-phase and quadrature band. All these
translations are done by using lookup tables, simplifying the implementation
while providing high-speed processing.

The transmit fifo shares the transmit chain with two other fifos for
automatic acknowledgment generation and rapid message forwarding. An access
control logic grants permission to a fifo if the fifo has data to transmit,
and the transmit chain is currently empty. Section \ref{subsec:ack} and
\ref{subsec:forward} will go into more details on these two fifos.

\begin{figure}[bht]
\centering
\includegraphics[width=0.98\columnwidth]{transmitting_arch}
\caption{IEEE 802.15.4 transmitter implementation. Three input FIFOs feed the
modulator implemented as lookup tables for byte to chip, and chip to half-sine
pulse mapping. This provides a compact and simple O-QPSK implementation with
support of automatic ACK generation and fast packet forwarding mechanisms.}
\label{fig:transmitting_chain}
\end{figure}

\subsection{Receiver}
 
\begin{figure}[h!]
\centering
\includegraphics[width=0.98\columnwidth]{receiving_arch}
\caption{IEEE 802.15.4 receiver implementation. Our implementation exploits
the fact that O-QPSK modulation can be decoded as MSK with adaptation of the
chip sequences. This simplifies the decoder implementation.}
\label{fig:receiving_arch}
\end{figure}

The implemented receiver chain (compare Figure \ref{fig:receiving_arch} resembles
the GNU Radio IEEE 802.15.4 receiver implementation described in
\cite{Schmid:802.15.4}. It exploits the fact that an O-QPSK modulated signal
with half-sine pulse shapes is equivalent to a minimum-shift keying (MSK)
signal, and thus a simpler frequency shift keying (FSK) demodulator can be
used to extract the chip sequence from the data stream \cite{cmos:rfic}. The
FSK-demodulator uses the in-phase and quadrature components coming from the RF
chip at 16~MS/s and produces an amplitude modulated (AM) signal, where the amplitude
corresponds to the frequency of the incoming signal. A matched filter
integrates the AM signal. The filter is sampled at the chip-frequency of
2~MS/s, indicating a logic 1 or 0. A preamble detection mechanism searches for
a 0x00 (a chip sequence of 64 bits) to identify incoming
packets. It also performs clock synchronization by occasionally adjusting the matched
filter. The receiver then calculates the Hamming distance $D$ between detected
symbols and the known chip sequences. The decoded symbol is the
index of the sequence that has the minimum distance $D$ from the sampled
sequence. The smaller the distance $D$ from a known chip sequence, the higher
the probability for a correct decoding. Once the symbol is decoded, the
receiver stores the decoded symbol in a \emph{RX fifo} and performs the frame
filtering. Various interrupts (detection of the start of frame delimiter SFD,
Length, Packet completed etc) are generated during the frame filtering,
informing the processor of the different events happening in the decoder.
These interrupts are crucial for timely processing inside the software
implementation of a MAC protocol.
Once a packet is fully received (e.g. when the processor receives the Packet Completed 
interrupt), the processor copies the packet from the RX fifo using a DMA
transfer into RAM for further processing.

\subsection{Radio and Automatic Gain Control}

A flexible SDR system needs fast and extensive control over the radio frontend
in order to be agile and adapt to a changing RF environment. The MAX2831
provides a 3-wire SPI interface for configuration, a 7~bit parallel bus for only gain
control, and an analog output providing the received signal strength. An
additional I/O line switches the frontend from receive to transmit, i.e., the
chip can not be used for full-duplex radio communication.

We offload as much of the radio control into the FPGA as possible. This 
simplifies the software to memory-mapped operations instead of dealing with a
stand alone SPI interface and incurring potential software interrupt
latencies. In addition, it speeds up the control loop and allows us to send
commands to the radio within $\mu$s. For example, the MAX2831 does not provide an
Automatic Gain Controller (AGC) built into the package. Thus, the FPGA has to
measure the RSSI, and adjust the radio gain through the SPI 
interface. The AGC loop tunes the gain setting within 5~$\mu$s without having
to involve the processor. This high speed AGC keeps tuning the gain setting to
prevent abrupt changes in signal strength and maintains a constant signal
amplitude.

\subsection{Hardware ACK Generation}\label{subsec:ack}

Many wireless protocols rely on acknowledgment messages (ACK) to ensure the
reliability of a transmission. According to IEEE 802.15.4, if a receiver
received a packet requesting an ACK, it has to generate an ACK frame and respond
to the transmitter in precisely 192 $\mu$s. Meeting this timing constraint can
be difficult for an SDR platform if they rely on the processor to software
decode the signal. Pure software decoding is favorable from a software
engineering's perspective, as more resources and code libraries are available.
In addition, the code can be written in a higher level language, such as C/C++
or Python. However, software decoding platforms must have enough computing
resources and low-latency interfaces between the processor and radio to meet
the stringent timing constraints. In most SDR systems, the interface between
the processor and the radio frontend is considered the data bottleneck.

Meeting those strict timing constraints has become a challenge to SDR
platforms. Nychis et al. characterize in \cite{2009.Nychis.nsdi} various
latencies from the Linux kernel to the Universal Software Radio Peripheral's
FPGA. Their results showed that it can take up to 9,000~$\mu$s with a mean
latency of 612~$\mu$s to get data from the Linux kernel to the USRP. Because
of this nondeterministic latency, the USRP is ill-suited for any
timing-constraint communication protocol. Nychis proposed several basic RF
blocks that if implemented close to the radio, can alleviate this problem.
However, it makes the system architecture significantly more complex to use as
this partition must be carefully planned.

MSR's SORA solves the interface issue by using a high bandwidth and
low-latency PCI-E interface, which achieves 360~ns one-way delay.
Additionally, a powerful multi-core processor with dedicated processing cores
allows real-time decoding and fast replies.

In contrast, the \sdr design uses hardware decoding, i.e., demodulation and
frame filtering are all performed in the FPGA. This drastically reduces
decoding latency and allows \sdr to quickly respond to incoming packets. In
\sdr, the ACK frame generation is also implemented in hardware. The ACK
generation block pre-stores preambles and SFD in a small 11-byte ACK fifo. The
only missing parts of the ACK are the data sequence number (DSN) and
corresponding FCS. If the incoming packet passes the CRC check, the hardware
copies the packet's DSN into the ACK fifo. At the same time, the ACK generator
block calculates the corresponding FCS and initiates a countdown for precise
timing. In our IEEE 802.15.4 implementation, the receiver calculates the CRC
and FCS within 12~$\mu$s, leaving 180~$\mu$s of slack for timing requirement.

\subsection{Radio Enhancements}
\label{subsec:enhance}

The receiver, transmitter, and hardware acknowledgment implementations are
everything needed to fully interact with commercially available IEEE 802.15.4
radios. The following sections describe enhancements to support more advanced
MAC features that are not always found in currently available radios. This
highlights the strength of \sdr to test and validate new radio architectures
that jointly optimize low-power MAC protocols.

\subsubsection{Multiple Address Recognition}

The most important decision a low-power MAC protocol makes is to stay awake or
go back to sleep if channel activity has been detected. Recent advances in
low-power MAC protocols showed that receiver initiated protocols have an
advantage over low-power listening types of protocols as they are more robust
against interfering noise \cite{2010.amac}. In a receiver initiated MAC
protocol, nodes that want to receive packets transmit a short probe. Nodes
that have packets to transmit buffer them locally until they hear a beacon
from the intended receiver. Once they decoded a beacon, the nodes acknowledge
it and start transmitting the data in the next packet. A-MAC \cite{2010.amac}
demonstrated that this can be done in a very efficient way, even if multiple
nodes try to send data to the same receiver. A-MAC exploits the fact that
acknowledgment messages will interfere constructively at the receiver. This
allows the receiver to make the crucial decision to stay awake, even if the
first data packet of the multiple transmitters collides, at which point a
back-off mechanism must resolve the conflict.

\begin{figure}[bht]
\centering
\includegraphics[width=0.98\columnwidth]{addr_recog_blk}
\caption{Packet reception and response in \sdr. This diagram shows the
paritioning of responsibilities between the hardware and software portions of
\sdr. The decision to support multiple address recognition requires software
support of the reception/response loop to keep hardware area costs reasonable.
In this tightly coupled model, the software controller takes advantage of the
hardware accelerated acknowledgments to meet the stringent latency
requirements of the 802.15.4 acknowledgment frame.}
\label{fig:addr_blk}
\end{figure}

Dutta et al. found in \cite{2010.amac} that a limiting factor of implementing
A-MAC is the address recognition mechanism of the radio. Hardware
acknowledgments can only be generated for ones own address. Thus, A-MAC
reprogrammed it's own address to the intended receiver's beacon address
(receiver's address with the MSB bit set) waiting for the beacon. This
introduced several problems and limits the protocol.
% Should we indicate at all what problems specifically this causes?

We implemented a multiple address recognition mechanism in \sdr. According to
the IEEE 802.15.4 standard, the address field can be as long as 8 bytes. This
leads to a design trade-off between implementing this functionality in
software or hardware by trading latency for FPGA area. We decided to have a
slight latency increase, and implemented a multiple address recognition
algorithm on the processor that can detect up to 8 addresses. However, this
latency increase does not impact the performance of the protocol, as \sdr can
read out the address faster from the receive fifo than the time it has to wait
to reply with an ACK. In addition, the multiple address recognition algorithm
takes advantage of the hardware acknowledgment mechanism preparing an ACK
while receiving the incoming packet. All the algorithm has to do is send a
signal to the peripheral letting it know to acknowledge this message.  Figure
\ref{fig:addr_blk} shows the state diagram of the algorithm and how the
hardware and software interact with each other. This demonstrates the
flexibility of the platform, allowing certain protocol functionalities that
are time critical to be done in hardware only (e.g.  hardware acknowledgments
generation) while others are implemented in software.


\subsubsection{Packet Forwarding}\label{subsec:forward}

\begin{figure}[ht]
\centering
\includegraphics[width=0.98\columnwidth]{fwd_blk}
\caption{Detailed packet forwarding state diagram. This is a more detailed
version of the packet forwarding mechanism shown on Figure \ref{fig:addr_blk}.
Packets are stored in an incoming FIFO, automatically incrementing the Relay
Count field and CRC as necessary. When a packet completes, it is automatically
forwarded if requested and its count is below the relay threashold.}
\label{fig:fwd_blk}
\end{figure}

Two basic services in wireless sensor networks are time synchronization and
flooding. The broadcast storm~\cite{syn:broadcast} is a common problem
flooding algorithms have to take care of, as retransmission of regular
broadcasts can amplify itself, and thus overload the network. Ferrari et al.
proposed {\em Glossy}~\cite{ferrari:efficient}, a fast flooding and time
synchronization mechanism. Glossy exploits the fact that in IEEE 802.15.4, if
multiple identical packets arrive within 0.5~$\mu$s at a receiver, they
collide constructively, and the receiver can successfully decode the message
with high probability. To achieve this, Glossy nodes forward a received packet
during a precise time slot and increment a relay counter in the packet.
Packets with identical relay counts will be transmitted at the same time, at
all nodes. Therefore, the packets will constructively interfere at the
receivers. The relay counter limits the maximum hop count of the packet,
making sure that packets don't live forever within the network. Glossy uses a
fragile software technique to guaranty software latencies. They have to
disable all interrupts and add no-operation (NOP) commands to adjust the
latencies during a Glossy transfer. Although the wireless links are lossy,
Glossy demonstrates 99.9999\% flooding reliability in a 10 hop deep network.

\"{O}sterlind et al. proposed in \cite{zero_copy} a zero-copy mechanism to
improve the throughput of multi-hop wireless sensor networks. He observes that
the critical path of zero-copy is the interface between radio and processor.
The packet forwarding is done by downloading the packet from the radio to the
processor, and uploading it back into the radio. Instead of
downloading and then uploading the same packet, it would be advantageous if
the radio stored the received packet and were able to immediately retransmit
it out. This method improves the total throughput to 97\% of the theoretical
upper bound.

We decided to implement a fast packet forwarding mechanism in the FPGA of the
\sdr platform. We use the data sequence number (DSN) as a relay counter.
Figure \ref{fig:fwd_blk} shows the state diagram. During reception of a
packet, the receiver stores the data in two different fifos, the receiver and
forwarding fifos. However, the two fifos store slightly different information
in the DSN and frame control sequence (FCS). The DSN gets automatically
incremented by 1, and the FCS is updated accordingly.

Using this mechanism, \sdr forwards a packet without interacting with the
software at all. This removes the critical path identified by \"{O}sterlind
and allows the retransmit to happen in less than 20~$\mu$s. This forwarding
also solves the fragile Glossy software implementation as no interrupts have
to be disabled on the software side, nor do we have to add any NOPs. In
addition, the forwarding can be precisely controlled, changing the latency to
any arbitrary, but fixed, delay (e.g. to 192~$\mu$s which would correspond to
the ACK turn around time). This would allow backward compatibility with some
existing commercial radios that have forwarding capabilities (e.g. the TI
CC2520), but have a transmit to receive turn around time of 192~$\mu$s as
specified by the IEEE 802.15.4 standard.

%- What can we do with this SDR platform?
%- E.g. IEEE 802.15.4 implementation. Why is that even interesting?
%  -- standards compliant == compliant with commercial radios
%  -- we built two radios (one basic FSK, one standards compliant)
%- Explain \& block diagram of TX and RX chain
%- Address Recognition
%- ACK generation
%- Message Forwarding
